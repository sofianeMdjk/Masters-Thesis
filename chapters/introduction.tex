\chapter*{Introduction} % Ne numérote pas le chapitre
\addcontentsline{toc}{chapter}{Introduction générale} % Référence l'introduction dans la table des matières
\chaptermark{Introduction générale}
L'intelligence artificielle (IA) vise à permettre aux machines d'émuler l'intelligence humaine. Cette branche de l'informatique inclut l’apprentissage, le raisonnement et l’auto-correction. L'intelligence artificielle, au fil des décennies, a révolutionné différents domaines en passant par la vision artificielle, les systèmes experts et le traitement automatique du langage naturel (TALN) afin de rendre les interactions entre l'homme et la machine plus naturelles. Depuis le début des années 1960s, plusieurs applications de TALN commencèrent à voir le jour à savoir, la traduction automatique, les systèmes de questions-réponses, mais aussi, les systèmes de reconnaissance de la parole. 

La reconnaissance de la parole a suscité un grand engouement que ce soit dans le domaine de la recherche ou dans le domaine multimédia. À ce jour, toutes les grandes compagnies comme Google, Apple, Facebook, Amazon ou Microsoft travaillent activement sur l'amélioration des technologies du traitement automatique de la parole afin d'offrir une meilleure interaction avec leurs produits. Cependant, une mauvaise reconnaissance peut mener à une mauvaise communication de l'information, une perte de temps et une frustration de l'utilisateur. Le monde arabe, et les arabophones plus particulièrement, ne disposent pas d'outils de reconnaissance de la parole aussi performants que ceux qui existent pour d'autres langues comme l'anglais. Ceci pose donc le défi d'améliorer ces outils pour la langue arabe. 

Il est clair que de nos jours, les systèmes de reconnaissance de la parole pour les langues étrangères, et notamment l'anglais, ont atteint des niveaux de fiabilité très intéressants. Ceci n'est malheureusement pas encore le cas pour la langue arabe qui présente des résultats nettement moins satisfaisants. Cette modeste performance est due au manque de corpus assez riches, mais aussi due à la complexité morphosyntaxique de la langue arabe par rapport aux autres langues. Le constat n'est cependant pas que négatif car, aujourd'hui, plusieurs équipes de recherches et entreprises travaillent sur l'amélioration de ces systèmes. Il y a eu également l'introduction de nouvelles techniques en matière d'apprentissage profond qui permettent de tirer plus facilement profit des corpus de dialogue en arabe et, ainsi, d'outrepasser la complexité de la langue. 
\newpage
Notre travail s'inscrit dans le cadre d'un projet de recherche. Dans l'optique de contribuer à l'amélioration des technologies du traitement automatique de la parole, nous avons développé un système de reconnaissance de la parole pour la langue arabe. Pour ce faire, nous avons commencé par étudier les concepts théoriques de ces systèmes ainsi que les différents travaux réalisés dans ce sens en examinant la littérature et l'état de l'art. Nous avons travaillé après cela sur la conception des différents modules du système pour passer ensuite à l'implémentation des différentes approches et techniques et faire une étude comparative des résultats obtenus. Nous avons enfin considéré les systèmes de questions-réponses comme domaine d'application. 

Comme contribution additionnelle à ce projet, nous avons mis en place un environnement de développement de systèmes de reconnaissance de la parole basé sur plusieurs architectures. Cet environnement permettra d'intégrer le module de reconnaissance de la parole à toute application de TALN. En plus de cela, il donnera la possibilité d'améliorer le système ou encore d'en créer un nouveau à l'aide de la panoplie d'outils que nous offrons.

Le premier chapitre de ce mémoire aura pour but de présenter les systèmes de reconnaissance automatique de la parole ainsi que les systèmes de questions-réponses. Le deuxième chapitre portera sur l'étude des approches et techniques utilisées pour le développement de ces systèmes. Cela nous permettra de mieux appréhender l'étude de l'état de l'art et de nous en inspirer. Le troisième chapitre quant à lui, sera le chapitre où nous concevons le système de reconnaissance de la parole, présentons les différents résultats obtenus ainsi qu'un cas d'application avec les systèmes de questions-réponses dans le quatrième chapitre. Nous concluons ce document avec une conclusion générale ainsi que les perspectives d'amélioration.
